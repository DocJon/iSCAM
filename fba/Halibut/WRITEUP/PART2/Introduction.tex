%!TEX root = /Users/stevenmartell/Documents/iSCAM-project/fba/Halibut/WRITEUP/Halibut.tex
\section{Introduction} % (fold)
\label{sec:introduction}

\textbf{Overarching objective:} Investigate the short-term and long-term consequences of adopting a smaller (26 inch or 66 cm) size limit on halibut spawning biomass, exploitable biomass, yield, and wastage.

Minimum size limits, or minimum weight limits, have been used in the commercial fishery since 1940.  In 1940, a minimum weight limit of 5 lb was used, which at the time corresponded to a fish of roughly 66 cm in length (or 26 inches).  In 1974 this minimum size limit was increased to 81.28cm or 32 inches in length.  Reasons for adopting a minimum size limit (or MSL hereafter) include conservation of juvenile halibut and increases in yield per recruit. Female halibut grow much faster than male halibut and recruit to the legal size at a much younger age than males.  The sex composition of the commercial catch is predominately females, and the age composition of landed males is much more uniform than the female fish \citep{Hare2012Rara}.  

Since at least 1996, the mean size-at-age of halibut in the setline survey has declined steadily over time.  Halibut seem to be experiencing slow than recent historical average growth rates.  Due to slower growth and a fixed 32 inch size limit, the age-at recruitment to the fishery should be shifting towards older females, and much older males.  Under a fixed exploitation rate policy, if the size-selectivity of the fishing gear captures fish below the minimum size, then it would be expected that the individual fish of sub-legal size would be capture more frequently when growth is slow.

The term wastage in the commercial setline fishery traditionally refers to fish that are captured by the fishing gear but not landed because the gear is either lost (something that occurred frequently during the days of the derby fishery), the fish is lost at the rail and dies, or the fish is returned to the ocean and dies.  In the directed commercial fishery, it is assumed that 16\% of the sublegal fish that are returned to the ocean die due to delayed mortality.  In the trawl fishery, discard mortality rates are assumed to be much higher (ca. 80-90\%).  

 In 2011, and estimated 2.2 million pounds of halibut were treated as wastage.  Assuming a modest value of \$5 per pound for fish in the 26-32 inch size category, this roughly equates to \$11 million dollars per year of halibut that are thrown overboard and assumed to have died.  Assuming a 16\% mortality rate, this would imply that roughly 13.75 million pounds were captured in the directed fishery and thrown overboard because they were of sublegal size.  Clearly there is a cost for halibut conservation (13.75 million pounds or \$68 million dollars assuming \$5 per pound) through the use of extra fuel and handling time to catch a given quota of legal size.

In the part of the report, a simulation model is used to evaluate the potential gains and losses of adopting a smaller size limit for the current directed fishery.  First a joint probability model for capturing and retaining a halibut of legal size is developed.  Followed by a series of simulations under alternative hypotheses about recruitment and future growth with the status quo size limit of 32 inches, a 29 inch size limit and a 26 inch size limit.  A series of performance metrics including future yield, wastage and the value of the catch and wastage is computed to evaluate the potential gains in yield and value in the directed setline fishery.

% section introduction (end)
