\documentclass{beamer}

%% STEP 1) SELECT THE THEM YOU WISH TO USE

\usetheme{Goettingen} % My favorite!
%\usetheme{PaloAlto}
%\usetheme{Hannover}
%\usetheme{Madrid} 
%\usetheme{Boadilla} % Pretty neat, soft color.
%\usetheme{default}
%\usetheme{Warsaw}
%\usetheme{Bergen} % This template has nagivation on the left
%\usetheme{Frankfurt} % Similar to the default 
%with an extra region at the top.
%\usecolortheme{seahorse} % Simple and clean template
%\usetheme{Darmstadt} % not so good
% Uncomment the following line if you want %
% page numbers and using Warsaw theme%
% \setbeamertemplate{footline}[page number]
%\setbeamercovered{transparent}
\setbeamercovered{invisible}
% To remove the navigation symbols from 
% the bottom of slides%
\setbeamertemplate{navigation symbols}{} 
%
\usepackage{graphicx}
\usepackage{array}
%\usepackage{bm}         % For typesetting bold math (not \mathbold)
%\logo{\includegraphics[height=0.6cm]{yourlogo.eps}}
%

%Custom Color theme
\usepackage{color}
\definecolor{bottomcolour}{rgb}{0.32,0.3,0.38}
\definecolor{middlecolour}{rgb}{0.08,0.08,0.16}
\setbeamerfont{title}{size=\Huge}
\setbeamercolor{structure}{fg=white}
\setbeamertemplate{frametitle}[default][center]

\setbeamercolor{normal text}{bg=black, fg=white}
\setbeamertemplate{background canvas}[vertical shading]
[bottom=bottomcolour, middle=middlecolour, top=black]

\setbeamertemplate{itemize item}{\lower3pt\hbox{\Large\textbullet}}
\setbeamerfont{frametitle}{size=\huge}


%end of color theme

%iscam logo
\newcommand{\iscam}{
	\raisebox{0.75ex}{$i$}%
	\textcolor{red}{\raisebox{0.25ex}{S}}%
	\textcolor{green}{\raisebox{0.00ex}{C}}%
	\textcolor{blue}{\raisebox{-.25ex}{A}}%
	\raisebox{-.50ex}{M}%
	}
\logo{\iscam}
%end of iscam logo

%Table of contents at begining of each section
\AtBeginSection[]
{
   \begin{frame}
       \frametitle{Outline}
       \tableofcontents[currentsection]
   \end{frame}
}
%end of table of contents

\title[\iscam]{Impacts of halibut bycatch and wastage on halibut coast-wide yield and spawning biomass}
\author{Steve Martell}
\institute[UBC]
{
University of British Columbia \\
\medskip
{\emph{martell.steve@gmail.com}}
}
\date{\today}
% \today will show current date. 
% Alternatively, you can specify a date.
%
\begin{document}
%
\begin{frame}
\titlepage
\end{frame}
%
\section{Introduction} % (fold)
\label{sec:introduction}
\begin{frame}[t]\frametitle{Objectives}
Provide an alternative investigation into the effects of halibut bycatch and wastage in the GOA and BSAI groundfish fisheries.\\
\pause \bigskip
\underline{Research Question:}\\
What are the impacts of \underline{bycatch} reductions on future estimates of halibut \textbf{biomass}, \textbf{yield}, \textbf{spawning biomass} and \textbf{wastage} by age--size categories over a fifteen year time horizon?\\
\pause \bigskip
\underline{To answer this question:}\\
Developed a deterministic sex/age-structured simulation model using IPHC assessment outputs to parameterize the model.
\end{frame}
% section introduction (end)

\section{Simulation model} % (fold)
\label{sec:simulation_model}
\begin{frame}[t,shrink]\frametitle{Simulation model overview}
	\begin{itemize}
		\item<+-| alert@+> 2 genders, 30 age-classes, 1996--2026 (or longer).
		\item<+-| alert@+> Inputs: IPHC initial age-composition, natural mortality (by sex), annual recruits, fishing mortality rates, size-selectivity, survey mean length-at-age.
		\item<+-| alert@+> Not used: empirical catch weight-at-age, age-smearing.
		\item<+-| alert@+> Exploitable biomass based on vulnerable biomass in commercial fishery.
		\item<+-| alert@+> Wastage based on joint probability: P(capture)*(1-P(retention))*$M_d$.
		\item<+-| alert@+> Apportionment based on 2011 apportionment.
		\item<+-| alert@+> Area based CEY's based area specific harvest rate.
		\item<+-| alert@+> Commercial fishery share: CEY - (other removals).
		\item<+-| alert@+> \alert{Reduction in bycatch translates to increase in commercial catch.}
	\end{itemize}
\end{frame}

\begin{frame}[t]\frametitle{Model outputs}
	\underline{Exploitable Biomass (EBio)}:\\
	Start of year biomass vulnerable to the commercial gear.\\
	\pause \medskip
	\underline{Spawning Biomass (SBio):}\\
	Start of year female mature biomass.\\
	\pause \medskip
	\underline{Commercial Yield (YBio):}\\
	Weight of commercial landings (excluding wastage).\\
	\pause \medskip
	\underline{Wastage (WBio):}\\
	Weight of dead discarded undersized fish in commercial fishery.\\
	\pause \medskip
	\underline{Lost Yield (LBio):}\\
	Difference between YBio with \emph{no} bycatch \& \emph{with} bycatch.\\
	\pause \medskip
	\underline{Yield Loss Ratio (YLR):}\\
	Ratio between the yield loss and the bycatch.\\
\end{frame}
% section simulation_model (end)
\section{Scenarios} % (fold)
\label{sec:scenarios}
\begin{frame}[m]\frametitle{Growth \& Recruitment Scenarios (states of nature)}
	\begin{enumerate}
		\item<1> Poor (60\% below average recruitment)
		\item<1> Average
		\item<1> Good (60\% above average recruitment)
	\end{enumerate}
	Growth: 
	\begin{enumerate}
		\item<2> Density-independent (using 2011 average length-at-age)
		\item<2> Density dependent. 
	\end{enumerate}
\end{frame}
%
\begin{frame}[m]\frametitle{Policy Scenarios}
	Three alternative policies:
	\begin{enumerate}
		\item Status quo: 2011 BSAI bycatch levels
		\item 50\% Reduction in BSAI bycatch (4ABCDE)
		 \begin{itemize}
		 	\item decrease from 5.535 million lb to 2.765 million lb.
		 \end{itemize}
		\item 50\% Reduction in GULF bycatch (3AB)
		\begin{itemize}
			\item decrease from 5.752 million lb to 2.876 million lb.
		\end{itemize}
	\end{enumerate}
\end{frame}
%
\begin{frame}[t]\frametitle{Summary plots}
	\centering
	\pgfimage[height=0.8\textheight]{../FIGURES/EBioDemo}
\end{frame}
%
% section scenarios (end)
\section{Results} % (fold)
\label{sec:results}
\subsection{EBio} % (fold)
\label{sub:ebio}
\begin{frame}[m]\frametitle{Decision table: EBio}
	Effect of bycatch reduction on Exploitable Biomass\\ \medskip
	\begin{tabular}{| b{1.1cm} |c|c|c|}
		\hline
		Policy & Status quo & 50\% BSAI  & 50\% GULF  \\
		\hline
		%
		DI Growth &
		\pgfimage[width=0.25\textwidth]{../FIGURES/fig_SQUO_DI_EBio} &
		\pgfimage[width=0.25\textwidth]{../FIGURES/fig_BSAI_DI_EBio} &
		\pgfimage[width=0.25\textwidth]{../FIGURES/fig_GULF_DI_EBio} \\
		\hline
		%
		DD Growth &
		\pgfimage[width=0.25\textwidth]{../FIGURES/fig_SQUO_DD_EBio} &
		\pgfimage[width=0.25\textwidth]{../FIGURES/fig_BSAI_DD_EBio} & 
		\pgfimage[width=0.25\textwidth]{../FIGURES/fig_GULF_DD_EBio} \\
		\hline
		%
		Average & 633 & 631.5 & 631 \\
		\hline
	\end{tabular}
\end{frame}
% subsection ebio (end)
%
\subsection{SBio} % (fold)
\label{sub:sbio}
\begin{frame}[m]\frametitle{Decision table: SBio}
	Effect of bycatch reduction on Female Spawning Biomass\\ \medskip
	\begin{tabular}{| b{1.1cm} |c|c|c|}
		\hline
		Policy & Status quo & 50\% BSAI  & 50\% GULF  \\
		\hline
		%
		DI Growth &
		\pgfimage[width=0.25\textwidth]{../FIGURES/fig_SQUO_DI_SBio} &
		\pgfimage[width=0.25\textwidth]{../FIGURES/fig_BSAI_DI_SBio} &
		\pgfimage[width=0.25\textwidth]{../FIGURES/fig_GULF_DI_SBio} \\
		\hline
		%
		DD Growth &
		\pgfimage[width=0.25\textwidth]{../FIGURES/fig_SQUO_DD_SBio} &
		\pgfimage[width=0.25\textwidth]{../FIGURES/fig_BSAI_DD_SBio} & 
		\pgfimage[width=0.25\textwidth]{../FIGURES/fig_GULF_DD_SBio} \\
		\hline
		%
		Average & 446 & 444 & 443.5 \\
		\hline
	\end{tabular}
\end{frame}
% subsection sbio (end)
%
\subsection{YBio} % (fold)
\label{sub:ybio}
\begin{frame}[m]\frametitle{Decision table: Commercial Yield}
	Effect of bycatch reduction on Commercial Yield\\ \medskip
	\begin{tabular}{| b{1.1cm} |c|c|c|}
		\hline
		Policy & Status quo & 50\% BSAI  & 50\% GULF  \\
		\hline
		%
		DI Growth &
		\pgfimage[width=0.25\textwidth]{../FIGURES/fig_SQUO_DI_YBio} &
		\pgfimage[width=0.25\textwidth]{../FIGURES/fig_BSAI_DI_YBio} &
		\pgfimage[width=0.25\textwidth]{../FIGURES/fig_GULF_DI_YBio} \\
		\hline
		%
		DD Growth &
		\pgfimage[width=0.25\textwidth]{../FIGURES/fig_SQUO_DD_YBio} &
		\pgfimage[width=0.25\textwidth]{../FIGURES/fig_BSAI_DD_YBio} & 
		\pgfimage[width=0.25\textwidth]{../FIGURES/fig_GULF_DD_YBio} \\
		\hline
		%
		Average & 101 & 103 & 103 \\
		\hline
	\end{tabular}
\end{frame}
% subsection ybio (end)
\subsection{Wastage} % (fold)
\label{sub:wastage}
\begin{frame}[m]\frametitle{Decision table: Comm. Wastage}
	Effect of bycatch reduction on Commercial Wastage\\ \medskip
	\begin{tabular}{| b{1.1cm} |c|c|c|}
		\hline
		Policy & Status quo & 50\% BSAI  & 50\% GULF  \\
		\hline
		%
		DI Growth &
		\pgfimage[width=0.25\textwidth]{../FIGURES/fig_SQUO_DI_WBio} &
		\pgfimage[width=0.25\textwidth]{../FIGURES/fig_BSAI_DI_WBio} &
		\pgfimage[width=0.25\textwidth]{../FIGURES/fig_GULF_DI_WBio} \\
		\hline
		%
		DD Growth &
		\pgfimage[width=0.25\textwidth]{../FIGURES/fig_SQUO_DD_WBio} &
		\pgfimage[width=0.25\textwidth]{../FIGURES/fig_BSAI_DD_WBio} & 
		\pgfimage[width=0.25\textwidth]{../FIGURES/fig_GULF_DD_WBio} \\
		\hline
		%
		Average & 2.494 & 2.581 & 2.589 \\
		\hline
	\end{tabular}
\end{frame}

% subsection wastage (end)
\subsection{Yield Loss Ratio} % (fold)
\label{sub:yield_loss_ratio}
\begin{frame}[m]\frametitle{Decision table: Yield Loss Ratio}
	Effect of bycatch reduction on Yield Loss Ratio\\ \medskip
	\begin{tabular}{| b{1.1cm} |c|c|c|}
		\hline
		Policy & Status quo & 50\% BSAI  & 50\% GULF  \\
		\hline
		%
		DI Growth &
		\pgfimage[width=0.25\textwidth]{../FIGURES/YLR/fig_SQUO_DI_ylr} &
		\pgfimage[width=0.25\textwidth]{../FIGURES/YLR/fig_BSAI_DI_ylr} &
		\pgfimage[width=0.25\textwidth]{../FIGURES/YLR/fig_GULF_DI_ylr} \\
		\hline
		%
		DD Growth &
		\pgfimage[width=0.25\textwidth]{../FIGURES/YLR/fig_SQUO_DD_ylr} &
		\pgfimage[width=0.25\textwidth]{../FIGURES/YLR/fig_BSAI_DD_ylr} & 
		\pgfimage[width=0.25\textwidth]{../FIGURES/YLR/fig_GULF_DD_ylr} \\
		\hline
		%
		Average & 2.494 & 2.581 & 2.589 \\
		\hline
	\end{tabular}
\end{frame}

% subsection yield_loss_ratio (end)
% section results (end)

\section{Summary} % (fold)
\label{sec:summary}
\begin{frame}[m,shrink]\frametitle{Summary}
	\begin{itemize}
		\item<+-| alert@+> 	Density-dependent growth dominates the scale of EBio in comparison to recruitment trends.\\
		\item<+-| alert@+>	Exploitable and spawning coastwide biomass are largely insensitive to BSAI and GULF bycatch.\\
		\item<+-| alert@+>	Fishing mortality is dominated by the directed commercial fishery.\\
		\item<+-| alert@+>	Reducing bycatch in BSAI or Gulf by 50\% ($\approx$ 2.7 million lb) results in a $\approx$ 2 million lb yr$^{-1}$ increase in the directed fishery.\\
	\end{itemize}
	\vspace{-1.5cm}
	\onslide<3>{
		\pgfimage[width=\textwidth]{../FIGURES/fig:FishRate.pdf}
	}
		
\end{frame}
% section summary (end)
\begin{frame}[m]\frametitle{Acknowledgments}
	IPHC staff\\
	At-sea Processors Association\\
	United Catcher Boats\\
	Pacific Seafood Processors Association\\
	Alaska Groundfish Data Bank\\
	Marine Conservation Alliance\\
	Groundfish Fourm\\
	Alaska Whitefish Trawlers\\
\end{frame}
% End of slides
\end{document}



























