%!TEX root=../Selex.tex
\section*{Discussion} % (fold)
\label{sec:discussion}

% Summary of the study
The over-arching objective of this simulation study was to determine if it is safe to assume more structural complexity in selectivity when in fact the real data come from a simple stationary process, and is it safer to assume simple structural complexity when real data come from a fishery with dynamic changes in selectivity.  To address this objective, a simulation model based with length-based selectivity and variable length-at-age by year was used to generate simulated data for four alternative assessment models that assumed: (a) selectivity was length-based and stationary, (b) selectivity was length-based and changed discretely in four time periods, and (c) selectivity was age-based and allowed to change each year, and (d) selectivity was age-based and interpolated over age and year using a bicubic spline and 60 equally spaced knots.  From the perspective of a n\"iave analyst who is unfamiliar with the history of the fishery and data, it may infact be safer to adopt a penalized likelihood approach to incorporate time-varying selectivity.


A more appropriate approach to specific case studies would be to develop a closed-loop feedback control system and an appropriate loss function to better elucidate which selectivity parameterization is more appropriate for achieving intended management objectives.  This is also known in the fisheries realm as management strategy evaluation \cite{MSE Crowds}.  Having an appropriate loss function to judge the performance of each alternative model would greatly improve model selection criterion from a policy performance perspective.

% section discussion (end)