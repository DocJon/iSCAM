%!TEX root=../Selex.tex
\section*{Discussion} % (fold)
\label{sec:discussion}

% Summary of the study
The over-arching objective of this simulation study was to determine if it is safe to assume more structural complexity in selectivity when in fact the real data come from a simple stationary process, and is it safer to assume simple structural complexity when real data come from a fishery with dynamic changes in selectivity.  To address this objective, a simulation model based with length-based selectivity and variable length-at-age by year was used to generate simulated data for four alternative assessment models that assumed: (a) selectivity was length-based and stationary, (b) selectivity was length-based and changed discretely in four time periods, and (c) selectivity was age-based and allowed to change each year, and (d) selectivity was age-based and interpolated over age and year using a bicubic spline and 60 equally spaced knots.  From the perspective of a n\"iave analyst who is unfamiliar with the history of the fishery and the source of the catch-age data, adopting a penalized time-varying selectivity may be more appropriate that assuming constant selectivity.  This general result is also consistent with a similar simulation study that examined time-varying changes in catchability \citep{wilberg2006performance}.

The addition of age-composition information into stock assessments greatly enhances the estimabiltiy of the underlying production function and related reference points \citep{magnusson2007mfd}.  In addition, age-composition information can also contribute to the estimation of over-all population scale via catch-curves and fixed assumptions about natural mortality and fisheries selectivity.  However, as assumptions about natural mortality and selectivity are relaxed (and freely estimated), information about population scaling degrades \citep{hilborn1992quantitative}. If the relative abundance index available for fitting lacks contrasting information to resolve confounding between overall productivity and population scale, the move towards more flexible selectivity models will exacerbate this problem.  In such cases where it is known that selectivity has changed over time, the addition of prior information on population scaling (i.e., priors for $B_o$ or survey $q$) will most likely be necessary. 

Earlier versions of the multivariate logistic likelihood for the age-composition data added a small fixed constant (1.e-30) to the observed and predicted age-proportions to ensure that the function remained defined when observed proportions-at-age were equal to 0 (i.e., this was done to avoid taking the natural logarithm of 0).  It turns out that the value of the fixed constant would have slight influences on the results and in some cases prevent the non-linear search routine from converging to a solution.  Adding a small constant to 0-observations that are likely to have high measurement error due is akin to imputing data and is probably not a safe practice in the long-run.  As an alternative approach, we adopted a method used by \cite{richards1997visualizing} where observed 0 proportions-at-age (or some minimum proportion, e.g., 2\% in their paper) were pooled with the adjacent year class for that year only.  For example, if the observed proportion of age-4 fish in 1985 was equal to 0, the estimation model would compute the likelihood for the number of age 4--5 fish in 1986;  there is no likelihood component for age-4 fish in 1985.  This pooling of year classes eliminates the need for adding small, potentially influential, constants to the likelihood. There is also a small caveat on this pooling approach, if a given cohort never appears in catch-age data (i.e., a complete year-class recruitment failure), the the estimation model with split the year class into the adjacent cohort.  

Changes in selectivity over time is also a special case of time-varying catchability.  It has already been demonstrated that additional sources of information, such as tagging data \citep{martell2002implementing}, area swept information \cite{winters1985interaction}, would reduce the confounding between stock size and stock productivity.  Statistical catch-at-age models rely on a separability assumption where year and age effects in the observed catch-at-age data can be partitioned into fishing mortality and selectivity, respectively.  Having axillary information on either one of these effects from area-swept estimate of relative fishing mortality or size-based selectivity based on tag return data would reduce potential confounding and improve the estimability of time-varying parameters \citep[e.g.,][]{sinclair2002disentangling}.

This simulation study examined the specific case where the true underlying selectivity is length-based and the corresponding age-based selectivity changes over time due to changes in growth rates.  Selectivity is a product of vulnerability and availability.  Vulnerability is the probability of catching a fish at a given time/location assuming the fish is available to harvest.  Availability is the probability of fish being present in the time/location where fishing activity is occurring.  These two processes are completely confounded and cannot be separated without additional information that directly measures either vulnerability or availability.  As a result of these two processes, the definition of fisheries selectivity may have many subtle differences among fisheries, or even among years in a given fishery.  Moreover, changes in harvest policy, or changes in allocation among regulatory areas, or fishing fleets, can result in dramatic changes in age-based selectivity due time/area interactions between various fishing fleets and stock distribution.  Given such complexities, it might be preferable to always adopt an age-based time-varying selectivity sub model in statistical catch-age assessments. Initial assessments could start with a very high penalty weights to constrain how much selectivity is allowed to vary (e.g., $\lambda^{(3)}$), then begin to relax the penalty and examine the sensitivity of both model fit, and policy performance of the subjective value.  Simply assuming a fixed selectivity model, or even block selectivities, is akin to extremely large penalty weights on time-varying selectivity.

One potential concern with the addition of more and more selectivity coefficients is that the assessment model begins to over-fit the age-composition data and explain the observed data with additional recruitment variation and more complex selectivity coefficients.  This result was observed in the Monte Carlo simulations in this study where the underlying data were generated with extremely complex selectivity patterns and the estimation model had a very flexible selectivity model with many estimated parameters.  Such an over-parameterized model would be of less utility in forecasting due to large uncertainties and confounding in selectivity and recruitment deviations in the terminal year.  Penalized likelihoods can ameliorate this to some extent, but we've also shown that the use of interpolation methods (e.g., bicubic splines) for computing age-specfic selectivity coefficients each year can perform well.  The subjective issue of importance in the case of using a bicubic spline is the number of spline knots that should be estimated for the year effect.  Presumably model selection could proceed in similar fashion as a stepwise-selection that was proposed by  \cite{thorson2012stepwise}.

 The vast majority of assessment models are age-based requiring age-specific estimates of fishing mortality rates, and hence age-based selectivity \citep{gavaris2002sif}.  Adopting a fixed age-based selectivity model would certainly lead to erroneous errors if the true underlying model is length-based and substantial changes in growth rates have occurred over time.  Two options for dealing with this problem are: (1) model length-based selectivity and using empirical length-age data, or (2) model age-based selectivity but allow selectivity to change over time.  The second option was recently adopted by the International Pacific Halibut Commission for dealing with changes in selectivity associated with changes in size-at-age and stock distribution \citep{stewart2012assessment}.  Selectivity in the directed Pacific halibut fishery is length-based as there are minimum size-limits in place.  The transition to time-varying age-based selectivity was adopted primarily because it solved a retrospective bias that has been of major concern for this stock in recent years.

% MOve to discussion
% It is fairly typical to see such lags in estimates of abundance, even in age-structured models \citep{walters2004simple,cox2008practical}

A more appropriate approach to specific case studies would be to develop a closed-loop feedback control system and an appropriate loss function to better elucidate which selectivity parameterization is more appropriate for achieving intended management objectives.  This is also known in the fisheries realm as management strategy evaluation \citep{de1986simulation,Cooke1999,smith1999implementing}.  Having an appropriate loss function to judge the performance of each alternative model would greatly improve model selection criterion from a policy performance perspective.

% section discussion (end)

\section*{Acknowledgments} % (fold)
\label{sec:acknowledgments}
	The authors would like to thank the organizers of the CAPAM workshop held on March 12-14, 2013 in La Jolla California.  The first author would also like to thank James Ianelli and Dave Fournier for the assistance in developing the bicubic spline model for this application.  Thanks to Andre Punt, Allen Hicks, Robyn Forrest, Ray Hilborn, James Thorson, and many other for feedback on earlier presentations of this work and the provision of the Pacific hake data.
% section acknowledgments (end)