	%
	%  untitled
	%
	%  Created by Martell on 2013-01-10.
	%  Copyright (c) 2013 UBC Fisheries Centre. All rights reserved.
	%
	\documentclass[12pt]{article}
	
	% Use utf-8 encoding for foreign characters
	\usepackage[utf8]{inputenc}
	
	% Setup for fullpage use
	\usepackage{fullpage}
	
	% Uncomment some of the following if you use the features
	%
	% Running Headers and footers
	%\usepackage{fancyhdr}
	
	% Multipart figures
	%\usepackage{subfigure}
	
	% More symbols
	%\usepackage{amsmath}
	%\usepackage{amssymb}
	%\usepackage{latexsym}

	%% -math- c/o jon schnute
	\newcounter{saveEq}
	\def\putEq{\setcounter{saveEq}{\value{equation}}}
	\def\getEq{\setcounter{equation}{\value{saveEq}}}
	\def\tableEq{ % equations in tables
	\putEq \setcounter{equation}{0}
	\renewcommand{\theequation}{T\arabic{table}.\arabic{equation}}
	\vspace{-5mm}
	}
	\def\normalEq{ % renew normal equations
	\getEq
	\renewcommand{\theequation}{\arabic{section}.\arabic{equation}}}

	\def\puthrule{ %thick rule lines for equation tables
	\hrule \hrule \hrule \hrule \hrule}

	
	% Surround parts of graphics with box
	\usepackage{boxedminipage}
	
	% Package for including code in the document
	\usepackage{listings}
	
	% If you want to generate a toc for each chapter (use with book)
	\usepackage{minitoc}
	
	% This is now the recommended way for checking for PDFLaTeX:
	\usepackage{ifpdf}

	% Bibliography
	\usepackage[round]{natbib}
	
	%\newif\ifpdf
	%\ifx\pdfoutput\undefined
	%\pdffalse % we are not running PDFLaTeX
	%\else
	%\pdfoutput=1 % we are running PDFLaTeX
	%\pdftrue
	%\fi
	
	\ifpdf
	\usepackage[pdftex]{graphicx}
	\else
	\usepackage{graphicx}
	\fi
	\title{Best practices for modeling time-varying selectivity}
	\author{ Steven Martell and Ian Stewart\\ International Pacific Halibut Commission\\ 2320 W Commodore Way, Suite 300,\\ Seattle WA, 98199-1287 }
	
	\date{2013-01-10}
	
	\begin{document}
	
	\ifpdf
	\DeclareGraphicsExtensions{.pdf, .jpg, .tif}
	\else
	\DeclareGraphicsExtensions{.eps, .jpg}
	\fi
	
	\maketitle
	
	
	\begin{abstract}
	% The following abstract is the place holder submitted for the selectivity workshop in La Jolla, CA

    Changes in the observed size- or age-composition of commercial catch can occur for a variety of reasons including: market demand, availability, temporal changes in growth, time-area closures, regulations, or change in fishing practice, to name but a few.  Two common approaches for dealing with time-varying selectivity in assessment models are the use of discrete time-blocks associated with an epoch in the history of the fishery, or the use of penalized random walk models for parametric or non-parametric selectivity curves.  Time block periods, or penalty weights associated with time-varying selectivity parameters, are subjective and often developed on an ad hoc basis. A factorial simulation-estimation experiment, with discrete or continuous changes in selectivity, is conducted to determine the best practices for modeling time-varying selectivity in fisheries stock assessments. Both the statistical properties of the assessment model and the policy implications of choosing the wrong model are taken into consideration.

	

	\end{abstract}
	
	
	\addcontentsline{toc}{section}{References}
	\bibliographystyle{apalike}
	\bibliography{/Documents/ARTICLES/Articles-1}
	\end{document}